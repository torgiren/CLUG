\documentclass[10pt]{beamer}
\usepackage{polski}
\usepackage[utf8]{inputenc}
\usepackage{verbatim}
\usepackage{listings}
\author{Marcin TORGiren Fabrykowski}
\institute{AGH - University of Science and Technology}
\title{PyRl - podstawy Pythona}
\usetheme{Warsaw}
\lstset{language=Python}
\begin{document}
\begin{frame}
	\titlepage
\end{frame}
\section{Podstawy składni}
\subsection{Pierwsze kroki}
\begin{frame}[fragile]
	\frametitle{Podstawy składni}
	\begin{block}<1->
		{Pierwszy program}
		\begin{lstlisting}[language=Python]
			print "Hello world"
		\end{lstlisting}
	\end{block}
	\begin{block}<2->
		{Drugi program}
		\begin{lstlisting}[language=Python]
			name=raw_input("Jak sie nazywasz? ")
			print "Witaj",name,"!"
		\end{lstlisting}
	\end{block}
\end{frame}
\begin{frame}[fragile]
	\begin{block}
		{Plik wykonywalny}
		\begin{lstlisting}[language=Python]
			#!/usr/bin/env python
			print "Hello World"
		\end{lstlisting}
	\end{block}
	\begin{block}
		{pamiętamy o uprawnieniach}
		\begin{lstlisting}[language=bash]
		chmod +x hello.py
		\end{lstlisting}
	\end{block}
\end{frame}
\subsection{Trochę o zmiennych}
\begin{frame}[fragile]
	\frametitle{Trochę o zmiennych}	
	\begin{block}
		{typy zmiennych}
		\begin{lstlisting}
		x="Python"
		print x
		x=10
		print x
		x=x*2
		print x
		\end{lstlisting}
	\end{block}
	\begin{block}
		{ciekawostka}
		\begin{lstlisting}
		x="haha "
		x*=5
		print x
		\end{lstlisting}
	\end{block}
\end{frame}
\subsection{Funkcje, pętle, ify}
\begin{frame}[fragile]
	\frametitle{Funkcje, pętle, ify}
	\begin{block}
		{Przykład funkcji}
		\begin{lstlisting}
			def Dzielniki(liczba):
			    if liczba<=1:
			        return "Podaj liczbe wieksza od 1"
			    for i in range(1,int(math.sqrt(liczba)+1)):
			        if not liczba%i:
			            print i
			            print liczba/i
			print Dzielniki(90)
			print Dzielniki(-10)
		\end{lstlisting}
	\end{block}
\end{frame}
\section{Kontenery}
\subsection{Listy}
\begin{frame}[fragile]
	\frametitle{Listy}
	\begin{block}
		{Lista}
		\begin{lstlisting}
linux=['Getnoo','Debian','Vista']
print linux
linux.append('Red Hat')
print linux
linux.remove('Vista')
for i in linux:
    print "Znam ",i
linux.append('Slackware')
if 'Slackware' in linux:
    print "Znam rowniez Slackware"
if 'Ubuntu' in linux:
    print "Znam Ubuntu"
else:
    print "Nie znam Ubuntu"
		\end{lstlisting}
	\end{block}
\end{frame}
\begin{frame}[fragile]
	\frametitle{Listy}
	\begin{block}<1->
		{Mamy 2 listy}
		\begin{lstlisting}
		x=['raz','dwa','trzy']
		y=['cztery','piec','szesc']
		\end{lstlisting}
	\end{block}
	\begin{block}<2->
		{Dodawanie list - append}
		\begin{lstlisting}
		x.append(y)
		print x
		\end{lstlisting}
		\uncover<3->{ŹLE!!!}
	\end{block}
	\begin{block}<4->
		{Dodawanie list - extend}
		\begin{lstlisting}
		x.extend(y)
		print x
		\end{lstlisting}
		\uncover<5->{DOBRZE!!!}
	\end{block}
\end{frame}
\subsection{Krotki}
\begin{frame}[fragile]
	\frametitle{Krotki}
	\begin{block}
		{Krotka}
		\begin{lstlisting}
		x=('1','2','3',4,5,6)
		\end{lstlisting}
	\end{block}
\end{frame}
\subsection{Słowniki}
\begin{frame}[fragile]
	\frametitle{Słowniki}
	\begin{block}
		{Słownik}
		\begin{lstlisting}
		x={'pies':'dog','dom':'house','okno':'window'}
		print x
		print x['pies']
		#print x['dog']
		print x.items()
		for (k,j) in x.items():
			print 'Polskie:",k," - angielskie:",j
		\end{lstlisting}
	\end{block}
\end{frame}
\section{Klasy}
\subsection{Pierwsza klasa}
\begin{frame}[fragile]
	\frametitle{Klasy}
	\begin{block}
		{Pierwsza klasa}
		\begin{lstlisting}
		class Zwierze:
		    def __init__(self):
		        self.imie='Benek'
		    def PiszImie(self):
		        print self.imie
		class Pies(Zwierze):
		    def DajGlos(self):
		        print "woof woof"
		a=Zwierze()
		a.PiszImie()
		#a.DajGlos()
		b=Pies()
		b.PiszImie()
		b.DajGlos()
		\end{lstlisting}
	\end{block}
\end{frame}
\end{document}

